Understanding the factors driving the emergence and spread of infectious diseases in an ever more globalised world is of utmost importance.
In recent years, scientists have explored the information contained in the genetic material -- DNA and RNA -- of pathogens (HIV, Influenza, Ebola, etc) to reveal the patterns of global spread of these disease-causing entities and the factors driving their emergence (climate, human behaviour, adaptation to new hosts, etc).
Making sense of all of the data, however, involves a lot of maths and computer science, and methodological innovation is being outpaced by the growth of data -- in both  size and quality.
The research in this thesis aims at bridging that gap between the data we have and the questions we would like to answer through the development of new statistical techniques to combine data and models.
I develop more efficient methods for (re)constructing the ancestry between organisms (\textit{phylogenetic trees}) which is an essential tool in the analysis of genomic data.
I also explore new ways of visualising data from computational analyses in order to aid scientists determine when they can trust their results.
Finally, I use modern statistical techniques to answer two important epidemiological questions about the 2013-2016 Ebola virus disease (EVD) epidemic in West Africa, the largest in history so far.
First, I investigate the factors that contributed to the epidemic and find that some regions that report no EVD cases were predicted to have high epidemic potential connected mostly with climatic factors such as seasonal temperature variation and rain and socio-economic factors such as the distance to large settlements.
The lack of overlap between areas with high predicted numbers of cases and persistence of the virus can explain why the epidemic did not spread further.
I then ask: did Ebola adapt to kill humans? While the answer is probably not, my findings teach us valuable lessons about the need to properly accommodate uncertainty in observational studies in order to make valid scientific statements.
Overall, the findings in this thesis demonstrate the potential of statistical methods in aiding epidemiological inference while also highlighting just how much we still need to learn.